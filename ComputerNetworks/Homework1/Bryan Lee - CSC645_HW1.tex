\documentclass[12pt]{article}
\usepackage{amsmath, amssymb, amsthm, graphicx, geometry, listings, xcolor, url, enumitem, fancyhdr, multirow}
\geometry{margin=1in}
\definecolor{darkgreen}{rgb}{0,0.5,0}
\pagestyle{fancy}

\lhead{SFSU, CSC 645-01}
\chead{Spring 2025}
\rhead{Computer Networks}

\title{Homework 1}
\author{Bryan Lee\\922649673}
\date{February 12, 2025}

\begin{document}

\maketitle
\thispagestyle{fancy}

\section*{Questions}

% Answer variables
\newcommand{\answerOne}{\textbf{
    There are many types of network end systems, or "hosts" that are connected
    computing devices, such as PCs, security cameras, printers, and etc.
}}

\newcommand{\answerTwo}{\textbf{
    A network protocol is a set of rules and constraints that defines how data/messages are transmitted and received
    across a network. It ensures that any device can communicate effectively, regardless of hardware, software, or architecture.
    Similar to a postal system, a protocol defines the structure of communication, including error handling, data format, and transmission procedures.
    Protocols also provide reliable data transfer and congestion control over a network.
    As the slides put it: "Protocols define the format, order of messages sent and recieved among network entities, and actions taken on message transmission, receipt."
}}

\newcommand{\physicalLayer}{\textbf{
    The physical layer is responsible for the physical transmission of data from one network element to an adjacent network element.
    It encodes the data into bits and then transmits the bits over the physical medium, i.e. fiber optics or Ethernet.
}}

\newcommand{\linkLayer}{\textbf{
    The link layer handles communication between adjacent network nodes (host or router).
    Link-layer services vary depending on the specific link-layer protocol.
    For example, reliable delivery from the transmitting node over one link to the receiving node is provided. 
    This differs from the reliable delivery service of TCP, which only offers reliable delivery from one end system to another rather than nodes.
    Some examples of link-layer protocols include Ethernet, WiFi, and the cable access network's DOCSIS protocol.
    Since datagrams typically need to traverse several links to travel from source to destination, they may be handled by different link-layer protocols en route.
    Overall, the link-layer protocol encapsulates the network datagram with a link-layer header to create a link-layer frame used for transportation.
}}

\newcommand{\networkLayer}{\textbf{
    The network layer is responsible for forwarding network-layer packets, known as datagrams, which are transport-layer segments encapsulated with a network-layer header from one host to another.
    While TCP and UDP function like postal mail, the network layer ensures the delivery of these segments to the transport layer at the destination host.
    One such well-known network layer protocol is the IP protocol, which defines the fields in the datagram as well as how the end systems and routers act on these fields.
    The network layer also includes routing protocols that determine the paths datagrams take between sources and destinations.
    Collectively, it is referred to as the IP layer, encompassing both the IP protocol and routing protocols.
}}

\newcommand{\transportLayer}{\textbf{
    The transport layer transports application-layer messages between application endpoints.
    On the Internet, the two main transport protocols are TCP and UDP.
    TCP provides a connection-oriented service to its application, mainly offering guaranteed delivery of application-layer messages to the destination,
    flow control, breaking long messages into shorter segments, and congestion control.
    On the other hand, UDP provides a connection-less service to its application.
    It is a no-frill service that provides no reliability, no flow control, and no congestion control for the benefit of reduced latency.
    Overall, the transport-layer protocol adds a transport-layer header to application-layer messages, forming a transport-layer segment,
    similar to giving the postal service a letter with a destination address.
}}

\newcommand{\applicationLayer}{\textbf{
    The application layer is where network applications and their application-layer protocols, including HTTP, SMTP, and FTP protocols, reside.
    It is distributed over multiple end systems, with the application in one system using the protocol to exchange packets of information (aka messages) with the application in another end system.
    Basically, it provides network services to end-user applications.
}}

\newcommand{\answerThree}{\textbf{
    The five layers in the Internet protocol stack is as follows:
    \begin{enumerate}[label=\arabic*.]
        \item \underline{physical:} \\ \physicalLayer \\
        \item \underline{link:} \\ \linkLayer \\
        \item \underline{network:} \\ \networkLayer \\
        \item \underline{transport:} \\ \transportLayer \\
        \item \underline{application:} \\ \applicationLayer \\
    \end{enumerate}
}}

% Main question prompt
\newcommand{\answerFour}{
    \textbf{To determine the time for the packet to propagate said distance, we can simply add the transmission delay with the propagation delay.}  
    \[
        d_{trans} = \frac{\text{packet length (bits)}}{\text{link transmission rate (bps)}}
        = \frac{1000 \times 8}{2 \times 10^6} = \text{0.004 second}
    \]

    \[
        d_{prop} = \frac{\text{length of physical link (m)}}{\text{propagation speed}}
        = \frac{2500 \times 10^3}{2.5 \times 10^8} = \text{0.01 second}
    \]

    \textbf{$\therefore$ The time for propagation is \( 0.014 \) seconds or \( 14 \) ms.}
}

\newcommand{\answerFive}{
    \textbf{Part 1: The total end-to-end delay for the packet expressed in terms of $d_i$, $s_i$, $R_i$ (i = 1,2,3), and L:}

    \[
        d_{E-E} = d_{proc} + d_{queue} + d_{trans} + d_{prop}
    \]

    \[
        = (2 \times d_{proc}) + (\frac{L}{R_1} + \frac{L}{R_2} + \frac{L}{R_3}) + 0
        + (\frac{d_1}{s_1} + \frac{d_2}{s_2} + \frac{d_3}{s_3})
    \]

    \textbf{Part 2: Computing the end-to-end delay:}

    \[
        \begin{aligned}
            d_{E-E} &= (2 \times (3 \times 10^{-3})) \\
            &\quad + \left(\frac{1500 \times 8}{2 \times 10^6} + \frac{1500 \times 8}{2 \times 10^6} + \frac{1500 \times 8}{2 \times 10^6} \right) \\
            &\quad + \left(\frac{5000 \times 10^3}{2.5 \times 10^8} + \frac{4000 \times 10^3}{2.5 \times 10^8} + \frac{1000 \times 10^3}{2.5 \times 10^8} \right) \\
            &= 0.006 + (0.006 + 0.006 + 0.006) + (0.02 + 0.016 + 0.004) \\
            &= \text{0.064 seconds = 64 ms}
        \end{aligned}
    \]
}

\newcommand{\answerSixPartOne}{
    \textbf{The throughput will be the bottleneck in the network: \\
    Throughput = min($R_1$, $R_2$, $R_3$) = 1 Mbps}
}

\newcommand{\answerSixPartTwo}{
    \[
        \frac{\text{File size (bits)}}{\text{Throughput (bps)}} = \frac{4 \times 10^6 \times 8}{1 \times 10^6} = \text{32 seconds}
    \]
}

\newcommand{\answerSeven}{
    \textbf{The transmission delay of the dedicated link:}

    \[
        \frac{\text{File size (bits)}}{\text{Throughput (bps)}} = \frac{40 \times 10^{12} \times 8}{100 \times 10^6}
        = \text{3200000 seconds} \approx \text{37 days}
    \]

    \textbf{Obviously with such a large transmission delay, I will be sending the data via FedEx overnight delivery.}
}

\begin{enumerate}
    \item Give three examples of network end systems. (10 Points) \\\\ \answerOne

    \item What is a network protocol? (10 Points) \\\\ \answerTwo

    \item What are the five layers in the Internet protocol stack? What are the principal
        responsibilities of each of these five layers? (15 Points) \\\\ \answerThree

    \item How long does it take a packet of length 1,000 bytes to propagate over a link of distance
        2,500 km, propagation speed \( 2.5x10^8 \) m/s, and transmission rate 2 Mbps? (10 Points) \\\\ \answerFour

    \item Consider a packet of length L which begins at end system A and travels over three links to
    a destination end system. These three links are connected by two packet switches. Let $d_i$, $s_i$, and $R_i$
    denote the length, propagation speed, and the transmission rate of link i, for i = 1, 2, 3.
    The packet switch delays each packet by $d_{proc}$ Assuming no queuing delays, in
    terms of $d_i$, $s_i$, and $R_i$ (i = 1,2,3)  and
    L, what is the total end-to-end delay for the packet?
    Suppose now the packet is 1,500 bytes, the propagation speed on all three links is $2.5x10^8$
    m/s, the transmission rates of all three links are 2 Mbps, the packet switch processing
    delay is 3 msec, the length of the first link is 5,000 km, the length of the second link is
    4,000 km, and the length of the last link is 1,000 km. For these values, what is the end-to-end delay? (25 Points) \\\\ \answerFive

    \item Suppose Host A wants to send a large file to Host B. The path from Host A to Host B has
    three links, of rates $R_1 = 1$ Mbps, $R_2 = 2$ Mbps, $R_3 = 4$ Mbps.
        \begin{enumerate}[label=(\alph*)]
            \item Assuming no other traffic in the network, what is the throughput for the file transfer? (10 Points) \\\\ \answerSixPartOne
            \item Suppose the file is 4 million bytes. Dividing the file size by the throughput, roughly
            how long will it take to transfer the file to Host B? (10 Points) \\ \answerSixPartTwo
        \end{enumerate}

    \item Suppose you would like to urgently deliver 40 terabytes (1 terabyte = \( 10^{12} \) bytes) data
    from Boston to Los Angeles. You have available a 100 Mbps dedicated link for data
    transfer. Would you prefer to transmit the data via this link or instead save the data in a
    hard drive and send using FedEx overnight delivery? Explain. (10 Points) \\
    \textit{Hint: Calculate the transmission delay of the dedicated link, and compare it with the delay
    when sending the data with FedEx overnight delivery.} \\\\ \answerSeven
    
\end{enumerate}


\end{document}